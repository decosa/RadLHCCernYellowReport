\section{Simulation of radiation environments}
\label{sec:simulation}

{\it Editors: I.~Dawson, S.~Mallows}  \\
{\it Contibuting authors: I.~Azhgirey, I.~Dawson, M.~Huhtinen, V.~Ivantchenko, D.~Kar, M.~Karacson, S.~Mallows, I.~Mandic, S.~Menke, P.~Miyagawa, A.~Di\,Mauro, A.~Mucha, S.~Pospisil, T.~Szumlak, V.~Vlachoudis}  \\

\noindent
Write introduction discussing: 
\begin{itemize}
\item The need of simulation for radiation background studies, starting with event generation followed by transport of particles in detector material. 
\item During design phase rely on radiation simulation predictions -- extrapolating from lower centre of mass collision energies. Challenge in determining uncertainties on the predictions. Safety factors?
\item Typical requirements of the experiments: 1 MeV neq fluence; ionising dose; hadrons $>$ 20 MeV; residual dose rates
\end{itemize}

\subsection{Event generation}
General discussion on event generator codes available for describing minimum bias $pp$ collisions. ATLAS uses PYTHIA8, CMS and LHCb use FLUKA embedded DPMJET. (Say something about ALICE, TOTEM, LHCf ?)

\subsubsection{PYTHIA}
\subsubsection{DPMJET}

~

\subsection{Particle transport codes}
General discussion on on particle transport codes used in the LHC experiments. ATLAS uses both FLUKA and GEANT4, CMS and LHCb use FLUKA.

\subsubsection{FLUKA \& FLUGG}
\subsubsection{MARS}
\subsubsection{GEANT3/CALOR}
\subsubsection{GEANT4}
~

\subsection{Simulation frameworks}
For example, ATLAS uses Git repository for shared geometry development -- shared with RP. Web tools. TWikis? 


\subsection{Fluence and dose predictions}

\subsubsection{ATLAS}

\subsubsection{CMS}

\subsubsection{LHCb}

\subsubsection{ALICE}

\begin{thebibliography}{99}
\end{thebibliography}

 
