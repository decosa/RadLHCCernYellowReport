\section{Introduction}
\label{sec:intro}

This report documents the knowledge and experience gained by the LHC experiments in running detector systems in harsh radiation environments, with a focus on inner detector systems. The deleterious effects of radiation is increasingly impacting detector operation and performance and measurements and observations have been made across the LHC experiments. It is timely to review the experiences of the LHC experiments and ask if the detector systems are operating and performing as expected? How accurate are the radiation damage models and predictions? Have there been unexpected effects? What mitigation strategies have been developed? Our understanding and modelling of radiation effects was originally tested in irradiation facilities, so strong motivation to cross check in-situ in the complex radiation fields of the LHC? \\

The goal of this report is to provide a major reference for future upgrades and for future collider studies, summarising the experiences and challenges of designing complex detector systems for operation in harsh radiation environments. In section \ref{sec:effects} we discuss the current knowledge on the effects of radiation on detector systems. In section \ref{sec:simulation} we describe how the radiation environments ...

%The interesting physics processes at the LHC are rare hence the need for high-luminosity and this is the source of the intense radiation fields, especially in the the inner detector regions close to the interaction point. 